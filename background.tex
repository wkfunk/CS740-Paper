\section{Background}
\subsection{Online Advertising}
In fiscal year 2016, internet advertising revenues in the United States totaled \$72.5 billion (a 21.8\% increase over 2015)~\cite{iabfy2016}.
Increasingly, much of this money is going to mobile advertising; in the first half of 2016, mobile surpassed search for the first time as the leading ad
format~\cite{iab2016}.
In such a massive and diverse industry, the technology surrounding the online ad ecosystem has grown exceedingly complex.
Online ads can be sourced via premium campaigns, ad networks, or ad exchanges and involve detailed targeting and accounting mechanisms~\cite{adscape}.
The number of players involved is constantly increasing, raising the barrier to disrupt the industry higher than ever ~\cite{lumascape}.

\subsection{Ad-blockers}
The growth of ad-blockers in recent years has been cause for alarm for many current online advertising stakeholders.
In a recent study of 2 million users, Malloy et al. found U.S. ad-blocker penetration to be 15.7-18.6\% (95\% CI)~\cite{malloy}.
An active measurement study by Pujol et al. estimated that 22\% of users were likely using ad-blockers~\cite{pujol}.
Surveys by various entities in the advertising industry have estimated that 26\% of desktop users~\cite{iab-blockers} and 37\% of mobile users~\cite{gwi} are blocking ads in the U.S.
Another industry survey estimates that 309 million are using mobile ad-blockers globally (16\% of global smartphone users)~\cite{pagefair}.
A simple back-of-the-envelope calculation reveals the scale of the impact this has on content providers: using the most conservative of the above ad-blocker estimates (15.7\%) and the \$72.5 billion spent in FY 2016, publishers, marketers, and others are missing out on approximately \$11.38 billion in revenue per year due to ad-blocking users.
We note that this can even be considered a conservative estimate, as a 2015 study by Adobe and PageFair estimated the cost to be \$20.3 billion in 2015, and predicted that it would grow to \$41.4 billion by 2016~\cite{pagefair2015cost}.
As both the quantity of money invested into digital advertising and number of ad-blocking users grow, this figure is becoming increasingly significant~\cite{iab2016}.

In response to this growth in ad-blocking technology, a new breed of partnerships are forming.
Eyeo GmbH, which makes Adblock Plus (installed on over 100 million active devices~\cite{abp}), began its Acceptable Ads program in 2011, which allows “large entities” to pay in order to have their ads whitelisted for its users.
As of 2015, this list was known to include Google, Microsoft, Amazon, and Taboola~\cite{whitelist}.
Increasingly, the battle over ad-blockers is being fought in the courtroom.
Recently, a German court banned Eyeo from collecting money from German publishing house Axel Springer~\cite{axel}.
These battles point to the higher-level rejection of advertisements by users, and suggest that a more sustainable compensation model is necessary for the long-term internet to continue to attract users and reward content producers.
