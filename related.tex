% 4) Related work - provide descriptions of the key research papers (at least 3) related to your proposed study
\section{Related Work}
As the rejection of advertisements by users and use of ad-blocking
software continues to grow worldwide, we must consider new models for
financing the Web as we know it. Past attempts at incorporating
micropayments into the Web include adding per-fee links to
HTML~\cite{w3c} and various forms of
crypto-currencies~\cite{peppercoin}. Recently, there have been a
number of (primarily industrial) attempts at solving this problem with
varying levels of success.

One such attempt is a model which participates in ad auctions on
behalf of the end user, such as Google's Contributor
program~\cite{contributor}, which is now discontinued, and
Atri~\cite{atri}. Though this approach will tend to find the market
value for space on a given page, it includes network and computational
overhead in order to participate in auctions before serving the
page. This limits the performance benefit of removing ads from
internet browsing, and also requires building and installing software
to participate in each ad exchange. Additionally, Google Contributor
was limited exclusively to advertisements served via Google AdSense,
which resulted in a user experience which was only partially
ad-free~\cite{adsense}.

Brave, a fork of the Firefox project, is a new browser which allows
users to divide up a monthly subscription to the sites that they visit
most often via Bitcoin-based micropayments~\cite{brave}. The project
uses the cryptographically secure Anonize protocol~\cite{anonize} to
prevent a user's browsing history from being linked to their
identity. Systems like these will face scalability concerns with
organizing payments when dealing with a large number of content
producers and consumers. Additionally, requiring the use of an
entirely different browser adds a barrier of entry for new users.

Sourcepoint~\cite{sourcepoint} and Blendle~\cite{blendle} offer consumers a
“content pass” which allows them access to a variety of content
providers for a monthly subscription. This approach is very similar to
the “paywall” which is currently implemented on many major publisher
sites, and will require cooperation from many large publishing players
in order to be feasible at a larger scale.

SatoshiPay~\cite{satoshi} and Autotip~\cite{autotip} allow users to
send micropayments in Bitcoin. However, Bitcoin transaction fees have
increased so much in recent years that micropayments are almost
infeasible.

Flattr Plus~\cite{flattrplus} (a partnership between Flattr and
Adblock Plus) allows consumers to distribute a monthly subscription
among selected content provides. Although the intersection of these
two user bases is quite large (ad-blocker users and those who want to
contribute to content providers), the implementation currently uses a
browser extension to track user behavior, leading to privacy concerns
