\RequirePackage[hyphens]{url}

% see https://www.usenix.org/conferences/author-resources/paper-templates
\documentclass[letterpaper,twocolumn,10pt]{article}
\usepackage{usenix,epsfig,endnotes}

\usepackage{url}
\usepackage{breakurl} 

\begin{document}

\date{}

% 1) Title - something that describes the nature of the project
\title{\Large \bf Exploring the Feasibility of a Scalable Internet Payment Architecture}

\author{
{\rm Ed Oakes} \\
{\rm \texttt{oakes@cs.wisc.edu}}
\and
{\rm Keith Funkhouser} \\
{\rm \texttt{wfunkhouser@cs.wisc.edu}}
}

\maketitle

% Ads
% Ad-blockers
% Auctions
% Current solutions

% 2) Abstract - provide a terse description of the problem you are addressing, why it is important and what you intend to do
\begin{abstract}
Online advertising revenue continues to climb higher each quarter. 
The growth of ad-blockers has been cause for alarm for many of the 
primary stakeholders in this complex ecosystem. Previous attempts 
have been made, unsuccessfully, to incorporate micropayments for 
content producers into the fabric of the internet. Here, we propose 
a project focusing on investigating the feasibility of a scalable, 
performant, and easy to use internet payment architecture.
\end{abstract}

% 3) Overview - similar to the abstract, provide a description of the motivation, challenges, RESEARCH QUESTION (which should be separated from the rest of the text and written in BOLD font), proposed approach (one paragraph) and expected outcomes.

\section{Overview}
Historically, online advertising has served as the primary source of
revenue for content creators. This framework has catalyzed the rapid
growth of the internet, as consuming quality content for free is
extremely attractive to users. However, producing such content is not
free for publishers who are currently left with two options: sell
space on their site to advertisers, or move to a subscription-based
model. Advertisements have largely won this battle, partially due to a
few key examples of the “paywall” architecture failing to retain
users~\cite{nyt}. This has led to a modern internet which is filled
with publishers entirely reliant on the ads that keep their lights on.

In this project, we aim to explore an alternative payment system,
which provides users with a choice to avoid advertisements on the
internet and still compensate publishers. Our focus in designing the
architecture lies in two key factors: convenience and scalability. The
ultimate goal is to minimize the burden on the user by providing a
payment system which integrates into the current internet ecosystem,
while designing an architecture flexible enough to scale with the
internet. Our ultimate research question is: \\
\\
\textbf{Can we devise a scalable internet payment architecture that
  supports ad-free browsing for users and fair compensation for
  content publishers?}

\section{Background}
\subsection{Online Advertising}
In the third quarter of 2016, U.S. advertisers invested \$17.6 billion
in digital advertising~\cite{iabq3}. Increasingly, much of this money
is going to mobile advertising; in the first half of 2016, mobile
surpassed search for the first time as the leading ad
format~\cite{iab2016}. In such a massive and diverse industry, the
technology surrounding the online ad ecosystem has grown exceedingly
complex. Online ads can be sourced via premium campaigns, ad networks,
or ad exchanges and involve detailed targeting and accounting
mechanisms~\cite{adscape}. The number of players involved is
constantly increasing, raising the barrier to disrupt the industry
higher than ever ~\cite{lumascape}.

\subsection{Ad-blockers}
The growth of ad-blockers in recent years has been cause for alarm for
many current online advertising stakeholders. In a recent study of 2
million users, Malloy et al. found U.S. ad-blocker penetration to be
15.7-18.6\% (95\% CI)~\cite{malloy}. An active measurement study by
Pujol et al. estimated that 22\% of users were likely using
ad-blockers~\cite{pujol}. Surveys by various entities in the
advertising industry have estimated that 26\% of desktop
users~\cite{iab-blockers} and 37\% of mobile users~\cite{gwi} are
blocking ads in the U.S. Another industry survey estimates that 309
million are using mobile ad-blockers globally (16\% of global
smartphone users)~\cite{pagefair}. A simple back-of-the-envelope
calculation reveals the scale of the impact this has on content
providers: using the most conservative of the above ad-blocker
estimates (15.7\%) and the \$17.6 billion spent in 2016 Q3,
publishers, marketers, and others are missing out on approximately
\$2.8 billion in revenue per quarter due to ad-blocking users. As both
the quantity of money invested into digital advertising and number of
ad-blocking users grow, this figure is becoming increasingly
significant~\cite{iab2016}.

In response to this growth in ad-blocking technology, a new breed of
partnerships are forming. Eyeo GmbH, which makes Adblock Plus
(installed on over 100 million active devices~\cite{abp}), began its
Acceptable Ads program in 2011, which allows “large entities” to pay
in order to have their ads whitelisted for its users. As of 2015, this
list was known to include Google, Microsoft, Amazon, and
Taboola~\cite{whitelist}. Increasingly, the battle over ad-blockers is
being fought in the courtroom. Recently, a German court banned Eyeo
from collecting money from German publishing house Axel
Springer~\cite{axel}. These battles point to the higher-level
rejection of advertisements by users, and suggest that a more
sustainable compensation model is necessary for the long-term internet
to continue to attract users and reward content producers.

% 4) Related work - provide descriptions of the key research papers (at least 3) related to your proposed study
\section{Related Work}
As the rejection of advertisements by users and use of ad-blocking
software continues to grow worldwide, we must consider new models for
financing the Web as we know it. Past attempts at incorporating
micropayments into the Web include adding per-fee links to
HTML~\cite{w3c} and various forms of
crypto-currencies~\cite{peppercoin}. Recently, there have been a
number of (primarily industrial) attempts at solving this problem with
varying levels of success.

One such attempt is a model which participates in ad auctions on
behalf of the end user, such as Google's Contributor
program~\cite{contributor}, which is now discontinued, and
Atri~\cite{atri}. Though this approach will tend to find the market
value for space on a given page, it includes network and computational
overhead in order to participate in auctions before serving the
page. This limits the performance benefit of removing ads from
internet browsing, and also requires building and installing software
to participate in each ad exchange. Additionally, Google Contributor
was limited exclusively to advertisements served via Google AdSense,
which resulted in a user experience which was only partially
ad-free~\cite{adsense}.

Brave, a fork of the Firefox project, is a new browser which allows
users to divide up a monthly subscription to the sites that they visit
most often via Bitcoin-based micropayments~\cite{brave}. The project
uses the cryptographically secure Anonize protocol~\cite{anonize} to
prevent a user's browsing history from being linked to their
identity. Systems like these will face scalability concerns with
organizing payments when dealing with a large number of content
producers and consumers. Additionally, requiring the use of an
entirely different browser adds a barrier of entry for new users.

Sourcepoint~\cite{sourcepoint} and Blendle~\cite{blendle} offer consumers a
“content pass” which allows them access to a variety of content
providers for a monthly subscription. This approach is very similar to
the “paywall” which is currently implemented on many major publisher
sites, and will require cooperation from many large publishing players
in order to be feasible at a larger scale.

SatoshiPay~\cite{satoshi} and Autotip~\cite{autotip} allow users to
send micropayments in Bitcoin. However, Bitcoin transaction fees have
increased so much in recent years that micropayments are almost
infeasible.

Flattr Plus~\cite{flattrplus} (a partnership between Flattr and
Adblock Plus) allows consumers to distribute a monthly subscription
among selected content provides. Although the intersection of these
two user bases is quite large (ad-blocker users and those who want to
contribute to content providers), the implementation currently uses a
browser extension to track user behavior, leading to privacy concerns

% 5) Methods - provide a description of the way in which you intend to address the research question.  This means describing the existing tools/systems that you will use, new tools or systems that you will build and the experiments that you will conduct.
\section{Methods}
The approach we have decided to take for the project is
system-building. Our task is to devise an internet payment
architecture which fulfills our guidelines of scalability and
maximizing usability. One of the major difficulties in this task is
addressing the shortcomings of the attempts we've outlined above.

Specifically, our plan is to explore a framework similar to that of
Google Contributor, so we will primarily need to address the concerns of
overhead and ad coverage. Any overhead required by our system will
limit its scalability, preventing widespread adoption from ever being
an option. Similarly, if our system is not able to extend to all forms
of digital advertising, ad-blocking users who are used to entirely
ad-free viewing will not be attracted to using it. Hence, we must find a
way to replace all advertisements (those from premium campaigns, ad
networks, and ad exchanges), rather than only those from one source.

There are also some practical questions which are left to be answered,
such as the feasibility of entering the advertising industry from a
business standpoint, but we will consider those outside the scope of
the project; our focus lies in design and implementation.

{\footnotesize \bibliographystyle{acm}
% 6) Bibliography - list related work using standard bibliographic form.  Simply listing titles and providing web links is not acceptable.
\bibliography{biblio}}

%\theendnotes

\end{document}